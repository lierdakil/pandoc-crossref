\begin{figure}
\centering
\pandocbounded{\includegraphics[keepaspectratio,alt={1}]{fig1.png}}
\caption{1}\label{fig:1}
\end{figure}

\begin{figure}
\centering
\pandocbounded{\includegraphics[keepaspectratio,alt={2}]{fig2.png}}
\caption{2}\label{fig:2}
\end{figure}

\begin{figure}
\centering
\pandocbounded{\includegraphics[keepaspectratio,alt={3}]{fig3.png}}
\caption{3}\label{fig:3}
\end{figure}

\begin{figure}
\centering
\pandocbounded{\includegraphics[keepaspectratio,alt={4}]{fig4.png}}
\caption{4}\label{fig:4}
\end{figure}

\begin{figure}
\centering
\pandocbounded{\includegraphics[keepaspectratio,alt={5}]{fig5.png}}
\caption{5}\label{fig:5}
\end{figure}

\begin{figure}
\centering
\pandocbounded{\includegraphics[keepaspectratio,alt={6}]{fig6.png}}
\caption{6}\label{fig:6}
\end{figure}

\begin{figure}
\centering
\pandocbounded{\includegraphics[keepaspectratio,alt={7}]{fig7.png}}
\caption{7}\label{fig:7}
\end{figure}

\begin{figure}
\centering
\pandocbounded{\includegraphics[keepaspectratio,alt={8}]{fig8.png}}
\caption{8}\label{fig:8}
\end{figure}

\begin{figure}
\centering
\pandocbounded{\includegraphics[keepaspectratio,alt={9}]{fig9.png}}
\caption{9}\label{fig:9}
\end{figure}

\begin{codelisting}

\caption{Listing caption 1}\label{lst:code1}

\begin{Shaded}
\begin{Highlighting}[]
\OtherTok{main ::} \DataTypeTok{IO}\NormalTok{ ()}
\NormalTok{main }\OtherTok{=} \FunctionTok{putStrLn} \StringTok{"Hello World!"}
\end{Highlighting}
\end{Shaded}

\end{codelisting}

\begin{codelisting}

\caption{Listing caption 2}\label{lst:code2}

\begin{Shaded}
\begin{Highlighting}[]
\OtherTok{main ::} \DataTypeTok{IO}\NormalTok{ ()}
\NormalTok{main }\OtherTok{=} \FunctionTok{putStrLn} \StringTok{"Hello World!"}
\end{Highlighting}
\end{Shaded}

\end{codelisting}

\begin{codelisting}

\caption{Listing caption 3}\label{lst:code3}

\begin{Shaded}
\begin{Highlighting}[]
\OtherTok{main ::} \DataTypeTok{IO}\NormalTok{ ()}
\NormalTok{main }\OtherTok{=} \FunctionTok{putStrLn} \StringTok{"Hello World!"}
\end{Highlighting}
\end{Shaded}

\end{codelisting}

\begin{codelisting}

\caption{Listing caption 4}\label{lst:code4}

\begin{Shaded}
\begin{Highlighting}[]
\OtherTok{main ::} \DataTypeTok{IO}\NormalTok{ ()}
\NormalTok{main }\OtherTok{=} \FunctionTok{putStrLn} \StringTok{"Hello World!"}
\end{Highlighting}
\end{Shaded}

\end{codelisting}

\begin{center}\rule{0.5\linewidth}{0.5pt}\end{center}

\begin{longtable}[]{@{}lll@{}}
\caption{\label{tbl:mytable}My table}\label{tbl:mytable}\tabularnewline
\toprule\noalign{}
a & b & c \\
\midrule\noalign{}
\endfirsthead
\toprule\noalign{}
a & b & c \\
\midrule\noalign{}
\endhead
\bottomrule\noalign{}
\endlastfoot
1 & 2 & 3 \\
4 & 5 & 6 \\
\end{longtable}

\begin{longtable}[]{@{}ll@{}}
\caption{\label{tbl:1}Table}\label{tbl:1}\tabularnewline
\toprule\noalign{}
a & b \\
\midrule\noalign{}
\endfirsthead
\toprule\noalign{}
a & b \\
\midrule\noalign{}
\endhead
\bottomrule\noalign{}
\endlastfoot
1 & 2 \\
\end{longtable}

\listoffigures

\listoftables

\listoflistings
